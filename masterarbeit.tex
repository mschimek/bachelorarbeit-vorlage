% Vorlage für eine Bachelorarbeit - 2012-2013 Timo Bingmann

% Dies ist nur eine Vorlage. Strikte Vorgaben wie die Bachelorarbeit auszusehen
% hat gibt es nicht. Darum können auch alle Teile angepasst werden.

\documentclass[12pt,a4paper,twoside]{scrartcl}

% Diese (und weitere) Eingabedateien sind in UTF-8
\usepackage[utf8]{inputenc}

% Verwende gute Type 1 Font: Latin Modern
\usepackage[T1]{fontenc}
\usepackage{lmodern}

% Sprache des Dokuments (für Silbentrennung und mehr)
\usepackage[german,english]{babel}

% Seitengröße - verwende fast die ganze A4 Seite
\usepackage[tmargin=22mm,bmargin=22mm,lmargin=20mm,rmargin=20mm]{geometry}

% Einrückung und Abstand zwischen Paragraphen
\setlength\parskip{\smallskipamount}
\setlength\parindent{0pt}

% Einige Standard-Mathematik Pakete
\usepackage{latexsym,amsmath,amssymb,mathtools,textcomp}

% Unterstützung für Sätze und Definitionen
\usepackage{amsthm}

\newtheorem{Satz}{Satz}[section]
\newtheorem{Definition}[Satz]{Definition}
\newtheorem{Lemma}[Satz]{Lemma}

\numberwithin{equation}{section}

% Deutsches Literaturverzeichnis
\usepackage{bibgerm}

% Unterstützung zum Einbinden von Graphiken
\usepackage{graphicx}

% Pakete die tabular und array verbessern
\usepackage{array,multirow}

% Kleiner enumerate und itemize Umgebungen
\usepackage{enumitem}

\setlist[enumerate]{topsep=0pt}
\setlist[itemize]{topsep=0pt}
\setlist[description]{font=\normalfont,topsep=0pt}

\setlist[enumerate,1]{label=(\roman*)}

% TikZ für Graphiken in LaTeX
\usepackage{tikz}
\usetikzlibrary{calc}

% Aktuelle Section und Untersection am Seitenkopf
\usepackage{fancyhdr}

\fancypagestyle{plain}{
  \fancyhead{}
  \fancyfoot{}
  \fancyfoot[LE,RO]{\normalsize\thepage}
  \renewcommand{\headrulewidth}{0pt}
  \renewcommand{\footrulewidth}{0pt}
}

\fancypagestyle{normal}{
  \setlength{\headheight}{20pt}
  \setlength\footskip{32pt}
  \fancyhead{}
  \fancyhead[LE]{\normalsize\textsc{\nouppercase{\leftmark}}}
  \fancyhead[RO]{\normalsize\textsc{\nouppercase{\rightmark}}}
  \fancyfoot{}
  \fancyfoot[LE,RO]{\normalsize\thepage}
  \renewcommand{\headrulewidth}{0.4pt}
  \renewcommand{\footrulewidth}{0pt}
}

% Hyperref für Hyperlink und Sprungtexte
\usepackage{xcolor,hyperref}

\hypersetup{
  pdftitle={Hier den Titel der Arbeit},
  pdfauthor={Hier den Author der Arbeit},
  pdfsubject={Stichworte, weiteres Stichwort},
  colorlinks=true,
  pdfborder={0 0 0},
  bookmarksopen=true,
  bookmarksopenlevel=1,
  bookmarksnumbered=true,
  linkcolor=blue!60!black,
  %linkcolor=black,
  citecolor=blue!60!black,
  urlcolor=blue!60!black,
  filecolor=green!60!black,
  pdfpagemode=UseNone,
  unicode=true,
}

% Paket zum Setzen von Algorithmen in Pseudocode mit kleinen Stilanpassungen
\usepackage[ruled,vlined,linesnumbered,norelsize]{algorithm2e}
\DontPrintSemicolon
\def\NlSty#1{\textnormal{\fontsize{8}{10}\selectfont{}#1}}
\SetKwSty{texttt}
\SetCommentSty{emph}
\def\listalgorithmcfname{Algorithmenverzeichnis}
\def\algorithmautorefname{Algorithmus}
\let\chapter=\section % repariert ein Problem mit algorithm2e

\begin{document}

%%%%%%%%%%%%%%%%%%%%%%%%%%%%%%%%%%%%%%%%%%%%%%%%%%%%%%%%%%%%%%%%%%%%%%

\pagestyle{empty} % keine Seitenzahlen

% Titelblatt der Arbeit
\begin{titlepage}

  \begin{center}\large

    \quad\includegraphics[height=17mm]{kit_logo_de.pdf} \hfill
    \includegraphics[height=20mm]{grouplogo-algo-blue.pdf}\quad\null

    \vfill

    Master Thesis
    \vspace*{2cm}

    {\huge Distributed String Sorting Algorithms \par}
    % Siehe auch oben die Felder pdftitle={}
    % mit \par am Ende stimmt der Zeilenabstand

    \vfill

    Matthias Schimek

    \vspace*{15mm}

    7 July 2019

    \vspace*{45mm}

    \begin{tabular}{rl}
      Betreuer: & Prof. Dr. Peter Sanders \\
      & Dipl. Inform. Zweiter Betreuer \\
    \end{tabular}
    
    \vspace*{10mm}

    Institut für Theoretische Informatik, Algorithmik \\
    Fakultät für Informatik \\
    Karlsruher Institut für Technologie

    % English:
    % Institute of Theoretical Informatics, Algorithmics \\
    % Department of Informatics \\
    % Karlsruhe Institute of Technology

    \vspace*{12mm}
  \end{center}

\end{titlepage}

%%%%%%%%%%%%%%%%%%%%%%%%%%%%%%%%%%%%%%%%%%%%%%%%%%%%%%%%%%%%%%%%%%%%%%

\vspace*{0pt}\vfill

\hrule\medskip

Hiermit versichere ich, dass ich diese Arbeit selbständig verfasst und keine anderen, als die angegebenen Quellen und Hilfsmittel benutzt, die wörtlich oder inhaltlich übernommenen Stellen als solche kenntlich gemacht und die Satzung des Karlsruher Instituts für Technologie zur Sicherung guter wissenschaftlicher Praxis in der jeweils gültigen Fassung beachtet habe.

\bigskip

\noindent
Ort, den Datum

% Unterschrift (handgeschrieben)

\vspace*{5cm}

\clearpage

%%%%%%%%%%%%%%%%%%%%%%%%%%%%%%%%%%%%%%%%%%%%%%%%%%%%%%%%%%%%%%%%%%%%%%

\vspace*{0pt}\vfill

\selectlanguage{english}
\begin{abstract}
\centerline{ Zusammenfassung}

Hier die deutsche Zusammenfassung

\end{abstract}

\vfill

\selectlanguage{english}
\begin{abstract}
\centerline{ Abstract}

\end{abstract}
\selectlanguage{english}

\vfill\vfill\vfill
\clearpage

%%%%%%%%%%%%%%%%%%%%%%%%%%%%%%%%%%%%%%%%%%%%%%%%%%%%%%%%%%%%%%%%%%%%%%

\vspace*{0pt}\vfill

\section*{Danksagungen}


\vfill\vfill\vfill
\clearpage

%%%%%%%%%%%%%%%%%%%%%%%%%%%%%%%%%%%%%%%%%%%%%%%%%%%%%%%%%%%%%%%%%%%%%%

\pagestyle{normal}
% markiere sections im Seitenkopf links und subsections rechts
\renewcommand\sectionmark[1]{\markboth{\thesection\quad\MakeUppercase{#1}}{\thesection\quad\MakeUppercase{#1}}}
\renewcommand\subsectionmark[1]{\markright{\thesubsection\quad\MakeUppercase{#1}}}

% Inhaltsverzeichnis
\tableofcontents

\clearpage

%%%%%%%%%%%%%%%%%%%%%%%%%%%%%%%%%%%%%%%%%%%%%%%%%%%%%%%%%%%%%%%%%%%%%%

\listoffigures
\listoftables
\listofalgorithms

\clearpage

%%%%%%%%%%%%%%%%%%%%%%%%%%%%%%%%%%%%%%%%%%%%%%%%%%%%%%%%%%%%%%%%%%%%%%

\section{Introduction}

\subsection{Motivation}

??

\subsection{Contribution}

First paper/thesis about distributed string sorting (Except for\cite{fischer2019lightweight}). Write about difference to simple algorithm in this paper, i.e. LCP-Computation, Prefix-Compression, LCP-Losertree.

New prefix-doubling approach (only computes permutation) \cite{sanders2013communication}



%%%%%%%%%%%%%%%%%%%%%%%%%%%%%%%%%%%%%%%%%%%%%%%%%%%%%%%%%%%%%%%%%%%%%%

\section{Preliminaries And Related Work}

\subsection{Definitions}

Some convention about PEs, (Multi-)sets of strings to sort.

Distinguishing Prefixes and their sum $D$, Longest Common Prefixes and their sum $L$. Total number of characters of the stringset $N$ 

\subsection{Related Work}

``simple'' distributed string sorting in \cite{fischer2019lightweight}

General notes about shared memory distributed string sorting (specific sections in \cite{bingmann2018scalable})
%%%%%%%%%%%%%%%%%%%%%%%%%%%%%%%%%%%%%%%%%%%%%%%%%%%%%%%%%%%%%%%%%%%%%%



\section{Distributed Mergesort}
In this section we describe our distributed \emph{Mergesort}-based approach to the following string sorting problem.

\begin{enumerate}
	\item \textbf{Input:} We sort a (multi-)set $S$ consisting of $n$ strings and $N$ characters. Each of the $p$ PE obtains either $\mathcal{O}(n/p)$ strings or a number of strings such that each PE has $\mathcal{O}(N/p)$ characters.
	\item \textbf{Output:} PE $i$ has a (multi-)set of strings $O_i$ with the following properties: \textcolor{red}{set might not be the right term}
	\begin{enumerate}
		\item The set $O_i$ is sorted locally.
		\item Let $i$ and $j$ be the indices of two PEs with $i < j$, then for all strings $s_i \in O_i$ and all strings $s_j \in O_j$ we find $s_i < s_j$.
	\end{enumerate}
\end{enumerate} 
The size and number of characters in the output sets $o_i$ depends on the splitting methods used (see \ref{merge_sort_sampling}).

The algorithm consists of the following building blocks:

\begin{enumerate}
	\item \textbf{Local sorting:} Each PE sorts its input set locally. We also determine the LCP-values of the local string set as a by-product.
	\item \textbf{Splitter sampling:} \label{merge_sort_sampling}Since the input set is already sorted, each PE can sample $s$ splitter equidistantly. From all $p \cdot s$ local samples $p-1$ splitters are globally determined and all local input sets are partitioned into $p$ buckets accordingly. 
	\item \textbf{String exchange:} The PEs perform an all-to-all exchange of the strings. On PE $i$ bucket $j$ determined in the previous step is sent to PE $j$. After this step all strings on PE $i$ are smaller than the strings on PE $j$ provided that $i$ < $j$.
	\item \textbf{Merging:} Although the strings are already globally sorted, as pointed out above, we still need to sort the received buckets on each PE. Since the buckets are already sorted we can use a multiway merge algorithm in this step.
\end{enumerate}

\subsection{Local Sorting}
Bingmann evaluated in his dissertation \cite{bingmann2018scalable} many sequential string sorting algorithms. We use an in-place radixsort variant as local sorter which is extended such that the LCP-values are computed while sorting with very small overhead \textcolor{red}{empirical evalutation needed for this assumption?}.

\begin{itemize}
	\item \textcolor{red}{say something about asymptotic run time}
\end{itemize}

\subsection{Splitter Sampling}
The goal of this step is to determine splitters such that the workload on each PE is nearly the same in the subsequent steps of the algorithm. When sorting atomic keys (i.e. integers or other fixed-length data) this correlates very well with the number of elements \textcolor{red}{citation needed?}. Thus, one tries to choose splitters such that about the same number of elements is distributed to each PE. Sorting strings differs in this respect. The main subsequent steps following the splitter selection are the all-to-all-string exchange and merging of the exchanged buckets. The runtime of the all-to-all-exchange is \textcolor{red}{look for alltoallexchange runtime} does not only depend on the number of exchanged strings but also on the size of these strings. On what concerns merging, it is not clear \textcolor{red}{merging evaluation, same number chars or strings}. 
This difference motivates the analysis of character-based sampling, i.e. sampling whose goal is to achieve nearly equal amounts of character on each PE after the all-to-all string exchange.

\subsubsection{String Based Sampling}
The \emph{String Based Sampling} approach aims at choosing splitters such that the number of strings on all PEs after the string exchange is equal. The local sampling is done as described in algorithm
\subsubsection{Character Based Sampling}




Overview: Algorithm \ref{alg:distributed-merge-sort} showing the ``buildig block'' of the distributed-multiway-mergesort. A simple enumeration + short explanation/description of the different ``buildig blocks" might be better? More details in the following subsections.

\begin{algorithm}
	\caption{Distributed-Multiway-Mergesort : General Algorithm}\label{alg:distributed-merge-sort}
	\SetKwFunction{DFS}{DFS}
	\SetKwFunction{sortLocally}{sort}
	\SetKwFunction{choseSplitter}{chooseSplitters}
	\SetKwFunction{partition}{partition}
	\SetKwFunction{exchange}{exchange}
	\SetKwFunction{merge}{merge}`
	\KwIn{On each PE: a multiset $s$ of strings with $|s| = n / p$}
	\BlankLine
	\sortLocally{$s$} \tcp*{sort locally}
	$splitters \leftarrow \choseSplitter{s}$ \tcp*{choose $p-1$ splitters}
	$s' \leftarrow \partition{s, splitters}$ \tcp*{partition $s$ into $p$ multisets $s'$}
	$s'' \leftarrow \exchange(s')$ \; 
	$s''' \leftarrow \merge(s'')$
	
	\KwOut{A sorted multiset of strings $s'''$ and its lcp values}
	
\end{algorithm}

Write something about runtime 

Write something about communication volume (dominated by the string-all-to-all exchange) should be in $\mathcal{O}(N)$


\subsection{String-Based Sampling}

splitter lengths are bounded by lcp-median.

\subsection{Character-Based Sampling}

splitter lengths are bounded by lcp-median.

\subsection{String All-To-All-Exchange}

No prefix compression.

Prefix compression using the precalculated lcp values. (Also mention ncessary decompression)

\subsection{Merging}

Merging using lcp-losertree. 

Short proof why we do not need to decompress the strings for the losertree?

\subsection{Decompression}

\subsection{Implementation Details}

\subsubsection{Sending Strings/Lcp values using MPI}

Different encoding layouts. One call for string and lcp values: Interleaved, sequential/ One call for strings and lcps values each.

\subsubsection{What else can I mention here?}

\section{Prefix-Doubling-Algorithm}

Overview: Algorithm \ref{alg:PrefixDoubling} showing the ``buildig blocks" of the prefix-doubling approach. Same as above: just a simple enumeration with a short description?

This algorithms only determines the \textcolor{red}{sorting permutation (better expression ?)}. 

Runtime more or less same as above + bloomfilter.

Important: Its communication volume is in $\mathcal{O}(D)$.
\begin{algorithm}
	\caption{Prefix-Doubling Approach: General Algorithm}\label{alg:PrefixDoubling}
	\SetKwFunction{DFS}{DFS}
	\SetKwFunction{sortLocally}{sort}
	\SetKwFunction{bloomfilter}{bloomfilter}
	\SetKwFunction{choseSplitter}{chooseSplitters}
	\SetKwFunction{partition}{partition}
	\SetKwFunction{exchange}{exchange}
	\SetKwFunction{merge}{merge}
	\KwIn{On each PE: a multiset $s$ of strings with $|s| = n / p$}
	\BlankLine
	\sortLocally{$s$} \tcp*{sort locally}
	$dPrefixes \leftarrow \bloomfilter(s)$ \tcp*{determine distinguishing prefix of each string}
	$splitters \leftarrow \choseSplitter{s}$ \tcp*{choose $p-1$ splitters}
	$s' \leftarrow \partition{s, splitters}$ \tcp*{partition $s$ into $p$ multisets $s'$}
	$s'' \leftarrow \exchange(s', dPrefix)$ \; 
	$s''' \leftarrow \merge(s'')$
	
	\KwOut{A sorted multiset of strings $s'''$ and its lcp values}
	
\end{algorithm}

\subsection{Prefix-Doubling Approach using Bloomfilter}

Describe the algorithm in more detail. What do I need to mention, where can/must I cite \cite{sanders2013communication} or/and \cite{schlag2013distributed}?

How much shall I write about golomb encoding? 

\Lemma[DistinguishinPrefixLength] {
	Let $d$ be the length of the distinguishing prefix of a string $s$ in a multiset $S$. By $d'$ we denote the value found with the prefix-doubling approach. Then, $d \le d' \le 2 \cdot d$ holds.
	\textcolor{red}{Assumption: No collision in hashfunction -> otherwise a probabilistic value}
}
\Lemma[Communication volume] {
	The communication volume of the algorithm is in $\mathcal{O}(D)$.
	\textcolor{red}{Assumption: No collision in hashfunction -> otherwise a probabilistic value}
}
\proof{
	
	A string $s$ participates in the prefix-doubling-algorithm until the depth is greater than or equal to its distinguishing prefix. In each step of the algorithm the depth is doubled. Hence, $s$ drops out of the process after at most $\lceil{\log(d)} \rceil$ rounds (starting with depth = 2). Therefore, at most $\lceil{\log(d)} \rceil$ hash values are sent for string $s$ in the whole process. Since $\lceil{\log(d)} \rceil \le d$, the total number of hash values sent sums up to at most $D$. It follows that the communication volume of the prefix-doubling algorithm is in $\mathcal{O}(D)$.
}

\subsection{All-To-All-Exchange using Prefix Compression and Distinguishing Prefix Length}

Clearly communication volume is in $\mathcal{O}(D')$. D' is the sum of the distinguishing prefix lengths determined by prefix-doubling. As shown above $D' \le 2 D$. 
Using Prefix Compression we can even do better. Should be something like $\mathcal{O}(D - L)$. Maybe I should avoid Landau-Notation since 
\[
n + L \le D \le 2L + n
\] (see \cite{bingmann2018scalable}).

\subsection{Implementation Details}


%%%%%%%%%%%%%%%%%%%%%%%%%%%%%%%%%%%%%%%%%%%%%%%%%%%%%%%%%%%%%%%%%%%%%%


	

%%%%%%%%%%%%%%%%%%%%%%%%%%%%%%%%%%%%%%%%%%%%%%%%%%%%%%%%%%%%%%%%%%%%%%

\section{Experimental Evaluation}
\subsection{Data}
\subsubsection{DToNGenerator}
Write something about the DToNGenerator.

%%%%%%%%%%%%%%%%%%%%%%%%%%%%%%%%%%%%%%%%%%%%%%%%%%%%%%%%%%%%%%%%%%%%%%

\section{Conclusion}

%%%%%%%%%%%%%%%%%%%%%%%%%%%%%%%%%%%%%%%%%%%%%%%%%%%%%%%%%%%%%%%%%%%%%%






\clearpage

%%%%%%%%%%%%%%%%%%%%%%%%%%%%%%%%%%%%%%%%%%%%%%%%%%%%%%%%%%%%%%%%%%%%%%

\bibliographystyle{gerplain}
\bibliography{literatur}

\end{document}
